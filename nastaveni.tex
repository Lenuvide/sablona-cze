% V tomto souboru se nastavují téměř veškeré informace, proměnné mezi studenty:
% jméno, název práce, pohlaví atd.
% Tento soubor je SDÍLENÝ mezi textem práce a prezentací k obhajobě -- netřeba něco nastavovat na dvou místech.

\usepackage[
%%% Z následujících voleb jazyka lze použít pouze jednu
  czech-english,		% originální jazyk je čeština, překlad je anglicky (výchozí)
  %english-czech,	% originální jazyk je angličtina, překlad je česky
  %slovak-english,	% originální jazyk je slovenština, překlad je anglicky
  %english-slovak,	% originální jazyk je angličtina, překlad je slovensky
%
%%% Z následujících voleb typu práce lze použít pouze jednu
  %semestral,		  % semestrální práce (nesází se abstrakty, prohlášení, poděkování) (výchozí)
  bachelor,			%	bakalářská práce
  %master,			  % diplomová práce
  %treatise,			% pojednání o disertační práci
  %doctoral,			% disertační práce
%
%%% Z následujících voleb zarovnání objektů lze použít pouze jednu
%  left,				  % rovnice a popisky plovoucích objektů budou zarovnány vlevo
	center,			    % rovnice a popisky plovoucích objektů budou zarovnány na střed (vychozi)
%
]{thesis}   % Balíček pro sazbu studentských prací


%%% Jméno a příjmení autora ve tvaru
%  [tituly před jménem]{Křestní}{Příjmení}[tituly za jménem]
% Pokud osoba nemá titul před/za jménem, smažte celý řetězec '[...]'
\author{Adam}{Černák}

%%% Identifikační číslo autora (VUT ID)
\butid{230049}

%%% Pohlaví autora/autorky
% (nepoužije se ve variantě english-czech ani english-slovak)
% Číselná hodnota: 1...žena, 0...muž
\gender{0}

%%% Jméno a příjmení vedoucího/školitele včetně titulů
%  [tituly před jménem]{Křestní}{Příjmení}[tituly za jménem]
% Pokud osoba nemá titul před/za jménem, smažte celý řetězec '[...]'
\advisor[Ing.]{Petr}{Petyovský}[Ph.D.]

%%% Jméno a příjmení oponenta včetně titulů
%  [tituly před jménem]{Křestní}{Příjmení}[tituly za jménem]
% Pokud osoba nemá titul před/za jménem, smažte celý řetězec '[...]'
% Nastavení oponenta se uplatní pouze v prezentaci k obhajobě;
% v případě, že nechcete, aby se na titulním snímku prezentace zobrazoval oponent, pouze příkaz zakomentujte;
% u obhajoby semestrální práce se oponent nezobrazuje (jelikož neexistuje)
% U dizertační práce jsou typicky dva až tři oponenti. Pokud je chcete mít na titulním slajdu, prosím ručně odkomentujte a upravte jejich jména v definici "VUT title page" v souboru thesis.sty.
\opponent[Ing.]{Tomáš}{Macho}[Ph.D.]

%%% Název práce
%  Parametr ve složených závorkách {} je název v originálním jazyce,
%  parametr v hranatých závorkách [] je překlad (podle toho jaký je originální jazyk).
%  V případě, že název Vaší práce je dlouhý a nevleze se celý do zápatí prezentace, použijte příkaz
%  \def\insertshorttitle{Zkác.\ náz.\ práce}
%  kde jako parametr vyplníte zkrácený název. Pokud nechcete zkracovat název, budete muset předefinovat,
%  jak se vytváří patička slidu. Viz odkaz: https://bit.ly/3EJTp5A
\title[System for measuring the residual volume of liquid in bottles]{Systém pro měření zůstatkového objemu kapaliny v láhvi}

%%% Označení oboru studia
%  Parametr ve složených závorkách {} je název oboru v originálním jazyce,
%  parametr v hranatých závorkách [] je překlad
\specialization[Automation and Measurement]{Automatizační a měřicí technika}

%%% Označení ústavu
%  Parametr ve složených závorkách {} je název ústavu v originálním jazyce,
%  parametr v hranatých závorkách [] je překlad
\department[Department of Control and Instrumentation]{Ústav automatizace a měřicí techniky}
%\department[Department of Biomedical Engineering]{Ústav biomedicínského inženýrství}
%\department[Department of Electrical Power Engineering]{Ústav elektroenergetiky}
%\department[Department of Electrical and Electronic Technology]{Ústav elektrotechnologie}
%\department[Department of Physics]{Ústav fyziky}
%\department[Department of Foreign Languages]{Ústav jazyků}
%\department[Department of Mathematics]{Ústav matematiky}
%\department[Department of Microelectronics]{Ústav mikroelektroniky}
%\department[Department of Radio Electronics]{Ústav radioelektroniky}
%\department[Department of Theoretical and Experimental Electrical Engineering]{Ústav teoretické a experimentální elektrotechniky}
%\department[Department of Telecommunications]{Ústav telekomunikací}
%\department[Department of Power Electrical and Electronic Engineering]{Ústav výkonové elektrotechniky a elektroniky}

%%% Označení fakulty
%  Parametr ve složených závorkách {} je název fakulty v originálním jazyce,
%  parametr v hranatých závorkách [] je překlad
%\faculty[Faculty of Architecture]{Fakulta architektury}
\faculty[Faculty of Electrical Engineering and~Communication]{Fakulta elektrotechniky a~komunikačních technologií}
%\faculty[Faculty of Chemistry]{Fakulta chemická}
%\faculty[Faculty of Information Technology]{Fakulta informačních technologií}
%\faculty[Faculty of Business and Management]{Fakulta podnikatelská}
%\faculty[Faculty of Civil Engineering]{Fakulta stavební}
%\faculty[Faculty of Mechanical Engineering]{Fakulta strojního inženýrství}
%\faculty[Faculty of Fine Arts]{Fakulta výtvarných umění}
%
%Nastavení logotypu (v hranatych zavorkach zkracene logo, ve slozenych plne):
\facultylogo[logo/FEKT_zkratka_barevne_PANTONE_CZ]{logo/UTKO_color_PANTONE_CZ}

%%% Rok odevzdání práce
\graduateyear{2025}
%%% Akademický rok odevzdání práce
\academicyear{2024/25}

%%% Datum obhajoby (uplatní se pouze v prezentaci k obhajobě)
\date{18.\,6.\,2025}

%%% Místo obhajoby
% Na titulních stránkách bude automaticky vysázeno VELKÝMI písmeny (pokud tyto stránky sází šablona)
\city{Brno}

%%% Abstrakt
\abstract[%
The bachelor thesis focuses on the design and implementation of a measurement system that streamlines existing methods for determining the residual liquid volume in a bottle for inventory purposes in hospitality operations (HoReCa). The proposed system determines the remaining volume of spirits indirectly - from the measured mass of the liquid and its density. It consists of a scale, a barcode reader, input output peripherals, and a processing unit containing a database and firmware with a graphical user interface. The work addresses the selection and interconnection of hardware components, the design of a database of essential bottle parameters, and the development of firmware that controls the entire process and presents the inventory results.
]{%
Bakalářská práce se zabívá návrhem a realizací měřicího systému pro zefektivnění dosavadních metod měření zbytkového objemu kapaliny v lahvi pro inventurní účely v hotelnických provozech (HoReCa). Navržený systém určuje zbytkový objem destilátu nepřímo – z naměřené hmotnosti kapaliny a její hustoty. Tvoří jej váha, čtečka čárového kódů, vstupně výstupní periferie a výpočetní jednotka obsahující databází a firmware s grafickým rozhraním. Práce řeší výběr a propojení hardwarových komponent, návrh databáze nezbytných parametrů lahví a vývoj firmwaru, který řídí celý proces a prezentuje výsledky inventury.}

%VERZE Č.2
%Bakalářská práce se zabývá návrhem a realizací měřicího systému pro zefektivnění dosavadních metod měření zbytkového objemu kapaliny v lahvi pro inventurní účely v hotelnických provozech (HoReCa). Objem se stanovuje nepřímo: z navážené hmotnosti kapaliny a  její hustoty. Měřicí systém zahrnuje váhu, čtečku čárových kódů, výpočetní jednotku a vstupně-výstupní periferie. Součástí řešení je výběr a propojení komponent, tvorba databáze klíčových údajů o lahvích a vývoj firmwaru s grafickým rozhraním, které řídí celý proces a prezentuje výsledky inventury.

%%% Klíčová slova
\keywrds[indirect liquid volume measurement, database, GUI, serial communication]{nepřímé měření objemu kapalin, databáze, GUI, sériová komunikace}
%databaze, GUI

%%% Poděkování
\acknowledgement{%
Rád bych poděkoval vedoucímu bakalářské práce
panu Ing. Petru Petyovskému, Ph.D.\ za odborné vedení,
konzultace, trpělivost a~podnětné návrhy k~práci.
}%