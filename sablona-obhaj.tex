% Soubory musí být v kódování, které je nastaveno v příkazu \usepackage[...]{inputenc}

\documentclass[%        Základní nastavení
  %draft,    				  % Testovací překlad
  12pt,       				% Velikost základního písma je 12 bodů
	t,                  % obsah slajdů bude vždy začínat od shora (nebude vertikálně centrovaný)
	aspectratio=1610,   % poměr stran bude 16:10 (všechny projektory v učebnách na Technické 12 Brno),
	                    % další volby jsou 43, 149, 169, 54, 32.
	unicode,						% Záložky a informace budou v kódování unicode
]{beamer}				    	% Dokument třídy 'zpráva', vhodná pro sazbu závěrečných prací s kapitolami
%\usepackage{etex}

\usepackage[utf8]		  % Kódování zdrojových souborů je v UTF-8
	{inputenc}					% Balíček pro nastavení kódování zdrojových souborů
	
\usepackage{graphicx} % Balíček 'graphicx' pro vkládání obrázků
											% Nutné pro vložení logotypů školy a fakulty

\usepackage[          % Balíček 'acronym' pro sazby zkratek a symbolů
	nohyperlinks				% Nebudou tvořeny hypertextové odkazy do seznamu zkratek
]{acronym}						
											% Nutné pro použití prostředí 'acronym' balíčku 'thesis'

%% Balíček hyperref je volán třídou beamer automaticky, proto není třeba následujícího kódu:
%\usepackage[
%	breaklinks=true,		% Hypertextové odkazy mohou obsahovat zalomení řádku
%	hypertexnames=false % Názvy hypertextových odkazů budou tvořeny
%											% nezávisle na názvech TeXu
%]{hyperref}						% Balíček 'hyperref' pro sazbu hypertextových odkazů
%											% Nutné pro použití příkazu 'nastavenipdf' balíčku 'thesis'

\usepackage{cmap} 		% Balíček cmap zajišťuje, že PDF vytvořené `pdflatexem' je
											% plně "prohledávatelné" a "kopírovatelné"

%\usepackage{upgreek}	% Balíček pro sazbu stojatých řeckých písmem
											%% např. stojaté pí: \uppi
											%% např. stojaté mí: \upmu (použitelné třeba v mikrometrech)
											%% pozor, grafická nekompatibilita s fonty typu Computer Modern!

%\usepackage{amsmath} %balíček pro sabu náročnější matematiky

\usepackage{booktabs} % Balíček, který umožňuje v tabulce používat
                      % příkazy \toprule, \midrule, \bottomrule


%%%%%%%%%%%%%%%%%%%%%%%%%%%%%%%%%%%%%%%%%%%%%%%%%%%%%%%%%%%%%%%%%
%%%%%%      Definice informací o dokumentu             %%%%%%%%%%
%%%%%%%%%%%%%%%%%%%%%%%%%%%%%%%%%%%%%%%%%%%%%%%%%%%%%%%%%%%%%%%%%

\input{nastaveni}      % v tomto souboru doplňte údaje o sobě, o názvu práce...
                       % (tento soubor je sdílený s textem práce)

%%%%%%%%%%%%%%%%%%%%%%%%%%%%%%%%%%%%%%%%%%%%%%%%%%%%%%%%%%%%%%%%%%%%%%%%

%%%%%%%%%%%%%%%%%%%%%%%%%%%%%%%%%%%%%%%%%%%%%%%%%%%%%%%%%%%%%%%%%%%%%%%%
%%%%%%     Nastavení polí ve Vlastnostech dokumentu PDF      %%%%%%%%%%%
%%%%%%%%%%%%%%%%%%%%%%%%%%%%%%%%%%%%%%%%%%%%%%%%%%%%%%%%%%%%%%%%%%%%%%%%
%% Při vloženém balíčku 'hyperref' lze použít příkaz '\pdfsettings'
\pdfsettings
%  Nastavení polí je možné provést také ručně příkazem:
%\hypersetup{
%  pdftitle={Název studentské práce},    	% Pole 'Document Title'
%  pdfauthor={Autor studenstké práce},   	% Pole 'Author'
%  pdfsubject={Typ práce}, 						  	% Pole 'Subject'
%  pdfkeywords={Klíčová slova}           	% Pole 'Keywords'
%}
\hypersetup{pdfpagemode=FullScreen}       % otevření rovnou v režimu celé obrazovky
%%%%%%%%%%%%%%%%%%%%%%%%%%%%%%%%%%%%%%%%%%%%%%%%%%%%%%%%%%%%%%%%%%%%%%%

\usetheme{VUT} 				% barvy a rozložení prezentace odpovídající VUT FEKT
% alternativně lze použít jiná berevná témata, ale bez záruky. Například: 
%\usetheme{Darmstadt} \usecolortheme{default2}
\logoheader					% vytvoření zkráceného loga VUT FEKT v hlavičce slajdu, nechte odkomentované



\begin{document}

% v případě zakomentování následujícího se zobrazí v pravém dolním rohu slajdů klikatelné navigační symboly 
\disablenavigationsymbols

% titulní snímek, vysazen bez horních, dolních a postranních lišt (volba plain),
% není tak vysazen ani nadpis snímku
\maketitle

%%%%%%%%%%%%%%%%%%%%%%%%%%%%%%%%%%%%%%%%%%%%%%%%%%%%%%%%%%%%%%%%%%%%%%%
% 1. snímek s cíli (zadaním) práce
%\begin{frame} 
%	% nadpis snímku
%	\frametitle{Cíle práce}
%	\begin{itemize}
%			\item Nastudovat
%			\item Popsat
%				\begin{itemize}
%					\item nastudované
%				\end{itemize}
%			\item Implementovat
%				\begin{itemize}
%					\item nastudované
%					\item nové
%				\end{itemize}
%			\item Porovnat a vyhodnotit
%				\begin{itemize}
%						\item výsledky
%				\end{itemize}
%	\end{itemize}
%\end{frame}

%1. Cíle práce
%2. Seznámení s problematikou ulohy.
%3. metody měření v současné době.. klady a zápory.
%4. Popis reseni, vcetne seznam a popisu jednotlivych komponent.
%5. Popis vzorové aplikace reseni.
%4. Závěr
%    - Shrnutí dosažených výsledků.
%    - Bakalářská práce.

\begin{frame}
	% nadpis snímku
	\frametitle{Cíle práce}
	\begin{enumerate}
			\item Nastudovat způsoby měření zůstatkového objemu kapalin za účelem inventury v HoReCa provozech.
			\item Porovnat dosavadní metody dostupné na trhu.
            %\item Návrh nového měřicího systému a pož
            \item Definovat požadavky na nově navržený systém
            \item Navrhněte blokové schéma celého systému a uveďte požadavky na jednotlivé komponenty.
            \item Volba vhodných komponent a jejich propojení s výpočetní jednotkou. Zprovoznění jednotlivých komponent.
			%	\begin{itemize}
			%		\item nastudované
			%	\end{itemize}
			%\item 
			%	\begin{itemize}
			%		\item nastudované
			%		\item nové
			%	\end{itemize}
			%\item Porovnat a vyhodnotit
			%	\begin{itemize}
			%			\item výsledky
			%	\end{itemize}
	\end{enumerate}
\end{frame}

%%%%%%%%%%%%%##################################
\begin{frame} %6
    \frametitle{Nově navrhovaný sytém} %BLOKOVE schmea
    
    \begin{figure}%	
				\centering
				%\vspace{0.5cm}	              % horizontální mezera
				\includegraphics[width=0.7\columnwidth]{obrazky/Blokové schéma.png}
    \end{figure}
\end{frame}

%%%%%%%%%%%%%##################################
\begin{frame} %3
    \frametitle{Dosavadní metody měření}
    \begin{itemize}
			\item Odměrné válce
				\begin{itemize}
					\item Obecné odměrné válce
                    \item Odměrné válce na alkohol
				\end{itemize}
			\item Váhy
	\end{itemize}

    \begin{figure}%	
				\centering
				\vspace{1cm}	              % horizontální mezera
				\includegraphics[width=0.2\columnwidth]{obrazky/odměrný válec.jpg}
                \includegraphics[width=0.3\columnwidth]{obrazky/odměrný válec na alkohol.jpg}
                \includegraphics[width=0.3\columnwidth]{obrazky/decibar.jpg}
    \end{figure}
\end{frame}

%%%%%%%%%%%%%##################################
\begin{frame} %4
    \frametitle{Výpočet objemu}  
    \begin{columns}[T]
    \begin{column}{0.6\textwidth}
      \begin{alertblock}{Lineární předpis funkce}
	   	$$ V = k \cdot m - q\, \left[\mathrm{m^3}\right] $$
	   \end{alertblock}
    
     \begin{block}{Výpočet směrnice}
	   	$$ k = \frac{V_{max}-V_{min}}{m_{max}-m_{min}}\, \left[\mathrm{m^3}/kg\right] $$
	   \end{block}
    
      \begin{block}{Výpočet offsetu}
	   	$$ q = V_{min} - k \cdot m_{min}\, \left[\mathrm{m^3}\right] $$
	   \end{block}
    \end{column}

 \begin{column}{0.4\textwidth}
 \vspace{0.5cm}
 \(V_{min}\) ...objem prázdné láhve

\(V_{max}\) ...objem plné láhve

\(m_{min}\) ...hmotnost prázdné láhve

\(m_{max}\) ...hmotnost plné láhve
\end{column}
 \end{columns}

\end{frame}

%%%%%%%%%%%%%##################################
%\begin{frame} %5
%    \frametitle{Výpočet objemu} %GRAFY
%    \begin{figure}%	
%				\centering
%				%\vspace{0.5cm}	              % horizontální mezera
%				\includegraphics[width=0.7\columnwidth]{obrazky/lineární funkce.png}
%    \end{figure}
%\end{frame}


%%%%%%%%%%%%%##################################
\begin{frame} %6
    \frametitle{Databáze}
    \vspace{1cm}
    \begin{table}[!h]
    \centering
    \begin{tabular}{|l|l|l|l|}
    \hline
    Název destilátu & Destilát č.1   & Destilát č.2   &  . . . \\ \hline
    EAN kód [-]&  &    &        \\ \hline
    Hmotnost prázdné láhve [g] &    &  &        \\ \hline
    Hmotnost plné láhve [g] &    &    &  \\ \hline
    Hmotnost víčka [g] &    &    &  \\ \hline
    Maximální objem láhve [l] &    &    &  \\ \hline
    Množství alkoholu [l] &    &    &  \\ \hline
    Obrázek [-] &    &    &  \\ \hline
    %Obrázek: obsahuje adresu/název obrázku, který je uložen ve složce
    \end{tabular}
    %\caption{Databáze destilátů}
    \end{table}
\end{frame}

%%%%%%%%%%%%%############################
\begin{frame} 
	\frametitle{Výběr komponent}
	
	\begin{columns}[T] 								% prostředí sloupce s umístěním nahoře
		\begin{column}{0.4\textwidth}		% první sloupec
			%Obrázek znázorňuje model:\\[2ex]
			%
			\begin{itemize}
				\item Mikrokontrolér
				\item Váha
				\item Čtečka čárového kódu
				\item Displej
			\end{itemize}
            \vspace{1cm}
            \includegraphics[width=0.7\columnwidth]{obrazky/gm65.PNG}
		\end{column}
		%
		\begin{column}{0.6\textwidth}		% druhý sloupec
			\begin{figure}%	
				\centering
				%\vspace{1cm}	              % horizontální mezera
				\includegraphics[width=0.7\columnwidth]{obrazky/raspberry-pi-4.png}
                \includegraphics[width=0.7\columnwidth]{obrazky/E3000.png}
				%lze vložit popisek, ale povetšinou je to v prezentaci zbytečné
				%\caption{Popisek obrázku}%
				%\label{obr:ukazka}
			\end{figure}
		\end{column}
	\end{columns}											% ukončení prostředí sloupce
\end{frame}

%%%%%%%%%%%%%##############################
\begin{frame} 
	\frametitle{Závěr}
    Jak budu pokračovat v bakalářské práci:
    \begin{itemize}
                \item Zprovoznit výsledný systém jako celek
				\item Vývoj firmwaru pro mikrokontrolér
				\item Vývoj databáze
			\end{itemize}
    
\end{frame}

%%%%%%%%%%%%#############################
% podekovani
\begin{frame}[c] 
% bez nadpisu snímku
	\frametitle{\mbox{ }}
	\begin{center}
		{\Huge Děkuji za pozornost!}
	\end{center}
\end{frame}
%%%%%%%%%%%%%%%%%%%%%

\begin{frame}[c] 
    \begin{table} [!h]
        \centering
        \begin{tabular}{|l|l|l|l|}
        \hline
             Název nalévače   & Nalévač č.1 & Nalévač č.2 &  . . .\\ \hline
             Výrobce &             &             &\\ \hline
             Hmotnost &             &             &\\ \hline
             Obrázek &             &             &\\ \hline
        \end{tabular}
        \caption{Databáze nalévačů}
        \label{tab:my_label}
    \end{table}

    \begin{figure}%	
				\centering
				%\vspace{1cm}	              % horizontální mezera
				\includegraphics[width=0.3\columnwidth]{obrazky/nalévač.jpg}
			\end{figure}
    
\end{frame}

\begin{frame}[c] 
    \frametitle{Zprovoznění komponent}

    \begin{figure}%	
				\centering
				%\vspace{1cm}	              % horizontální mezera
				\includegraphics[width=0.6\columnwidth]{obrazky/nastaveni putty.png}
			\end{figure}
    
\end{frame}

\begin{frame}[c] 
    \frametitle{Zprovoznění komponent}

    \begin{figure}%	
				\centering
				%\vspace{1cm}	              % horizontální mezera
                \includegraphics[width=0.7\columnwidth]{obrazky/Testování komunikace pomocí PuTTY.jpg}
			\end{figure}
    
\end{frame}

\begin{frame}[c] 
    \frametitle{Zprovoznění komponent}

    \begin{figure}%	
				\centering
				%\vspace{1cm}	              % horizontální mezera
                \includegraphics[width=0.8\columnwidth]{obrazky/DS1Z_QuickPrint6.png}
			\end{figure}
    
\end{frame}


\begin{frame}[c] 
    \frametitle{Váha - výstupní data (RS-232)}

    \begin{itemize}
                \item 1 start bit
				\item 8 datových bitů
				\item 1 stop bit
                \item Bez paritního bitu
    \end{itemize}
    \vspace{1cm}	
    
    Data reprezentována jako ASCII:    
    \begin{figure}%	
				\centering
				%\vspace{1cm}	              % horizontální mezera
				\includegraphics[width=0.9\columnwidth]{obrazky/ascii.png}
			\end{figure}
    
\end{frame}

%%%%%%%%%%%%%
%\begin{frame} 
%	\frametitle{Klíčové nástroje}
%
%	% prostředí 'alertblock', které slouží pro zdůraznění informace
%	\begin{alertblock}{Pro práci je klíčový Eulerův vzorec}
%		$$\eul^{\jmag x}=\cos x + \jmag\sin x$$
%	\end{alertblock}
%
%	\vspace{4ex}
%	Eulerova identita je speciálním případem tohoto vzorce, jestliže dosadíme $x=\uppi$\,:
%
%	% prostředí 'block', které slouží jako informativní
%	\begin{block}{Eulerova identita}
%		$$\eul^{\jmag \uppi}=\cos \uppi + \jmag\sin \uppi,$$\\
%		odkud vyplývá
%		$$\eul^{\jmag \uppi}+1=0.$$
%	\end{block}
%\end{frame} 

%%%%%%%%%%%%%
%\begin{frame} 
%	\frametitle{Plošný spoj}
%	
%	\begin{columns}[T] 								% prostředí sloupce s umístěním nahoře
%		\begin{column}{0.4\textwidth}		% první sloupec
%			Obrázek znázorňuje model:\\[2ex]
%			%
%			\begin{itemize}
%				\item Deska
%				\item Součástky
%				\item Signály
%				\item Napájení
%			\end{itemize}
%		\end{column}
%		%
%		\begin{column}{0.6\textwidth}		% druhý sloupec
%			\begin{figure}%	
%				\centering
%				\vspace{1cm}	              % horizontální mezera
%				\includegraphics[width=0.8\columnwidth]{obrazky/soucastky}
%				%lze vložit popisek, ale povetšinou je to v prezentaci zbytečné
%				%\caption{Popisek obrázku}%
%				%\label{obr:ukazka}
%			\end{figure}
%		\end{column}
%	\end{columns}											% ukončení prostředí sloupce
%\end{frame}

%%%%%%%%%%%%%
%\begin{frame} 
%	\frametitle{Výsledky}
%	\vspace{1cm}
%	\begin{table}[]
%		\centering
%		\caption{Výsledky měření mobilních sítí}
%		\label{tab:tabulka}
%			\begin{tabular}{lcc}
%				\toprule
%					Technologie  & Rychlost stahování [kB/s] & Rychlost nahrávání [kB/s] \\
%				\midrule
%					GPRS (2,5G)	& 7,2 	& 3,6\\
%					UMTS 3G     & 48 		& 48\\
%					HSPA (3,5G)	&	1\,706	&	720\\
%					LTE (4G) 		& 40\,750 & 10\,750\\
%				\bottomrule                                       
%			\end{tabular}
%	\end{table}
%\end{frame}


% otázky oponenta
%\frame{
%\frametitle{Otázky oponenta}
%	\emph{Jaká je souvislost Vašeho vzorce (1.2) s~Maxwellovými rovnicemi v~integrálním tvaru?}\\[2ex]
%	%
%	Již staří Římané\,\dots
%}

\end{document}
