Bakalářská práce
bakalářský studijní program Automatizační a měřicí technika
Ústav automatizace a měřicí techniky

NÁZEV TÉMATU:
Systém pro měření zůstatkového objemu kapaliny v láhvi
POKYNY PRO VYPRACOVÁNÍ:
Cílem práce je navrhnout a implementovat systém pro měření zůstatkového objemu kapaliny v lahvích za účelem
inventury v HoReCa provozech. Systém bude navržen jako modulární sestávající z digitální váhy, čtečky
čárových kódu, modulu s výpočetní jednotkou, displeje a ovládacích prvků.

1. Seznamte se s principy a vlastnostmi systémů pro měření zůstatkového objemu kapaliny v lahvích za účelem
inventury v HoReCa provozech.
2. Diskutujte možnosti stávajících řešení dostupných na trhu.
3. Definujte detailní požadavky na nově navrhovaný systém.
4. Navrhněte blokové schéma celého systému a uveďte požadavky na jednotlivé komponenty (váha, čtečka
kódů, mikrokontroler, displej).
5. Zvolte vhodné komponenty, navrhněte jejich vhodné propojení s výpočetní jednotkou. Jednotlivé komponenty
navrhovaného systému zprovozněte.
6. Zprovozněte výsledný systém jako celek a navrhněte firmware pro mikrokontroler.
7. Na vhodném příkladu odzkoušejte, ověřte a demonstrujte funkčnost celého systému.
8. Zhodnoťte dosažené výsledky a navrhněte další možná vylepšení.

DOPORUČENÁ LITERATURA:
[1] EVERETT, H.,R.: Sensors for Mobile Robots theory and application, CRC Press 1995, ISBN 1568810482.
[2] Raspberry Pi Documentation. Raspberry Pi Foundation. [online]. August 30. 2023. Dostupné na WWW:
<https://www.raspberrypi.com/documentation/computers/>.

Fakulta elektrotechniky a komunikačních technologií, Vysoké učení technické v Brně / Technická 3058/10 / 616 00 / Brno