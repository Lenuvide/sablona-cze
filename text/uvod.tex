\chapter*{Úvod}
\phantomsection
\addcontentsline{toc}{chapter}{Úvod}

Tato semestrální práce se věnuje návrhu systému pro měření zůstatkového objemu kapalin pro ulehčení inventurních činností v HoReCa(Hotel/Restaurant/Café) podnicích. Běžná fyzická inventura kapalin, v našem případě destilátů, obnáší přelévání alkoholu do odměrných válců a zpět pro zjištění jejich objemu. Tato metoda je časově náročná a navíc dochází k dalším nežádoucím jevům jako dehonestaci alkoholu, plýtvání vody, atd.
%Nově navržený systém má za úkol z hmotnosti kapaliny vypočíst jeho objem.

Nově navržený měřicí systém se bude skládat z mikrokontroléru \cite{Raspberry pi}, váhy, čtečky čárového kódu \cite{Sensor for robots}, vstupních a výstupních periferií, které jsou navzájem propojené. Jeho úkolem je z naměřené hmotnosti lahve a jejího objemu vypočítat výsledný objem bez nutnosti vůbec láhev otevírat.
Při dosavadních metodách měření může určení objemu jedné láhve trvat až 2 minuty. Cílem nového systému je tuto dobu omezit na odhadovaných 3 - 5 sekund. Běžně při inventurách měříme i 30 lahví a pro tyto případy má smysl implementovat nově navrhovaný systém.

%Tato práce obsahuje v 1. polovině teoretickou část pro popis jednotlivých komponent, specifikování požadavků na nich






































%Úvod studentské práce, např...
%
%Nečíslovaná kapitola Úvod obsahuje \uv{seznámení} čtenáře s~problematikou práce.
%Typicky se zde uvádí:
%(a) do jaké tematické oblasti práce spadá, (b) co jsou hlavní cíle celé práce a (c) jakým způsobem jich bylo dosaženo.
%Úvod zpravidla nepřesahuje jednu stranu.
%Poslední odstavec Úvodu standardně představuje základní strukturu celého dokumentu.
%
%Tato práce se věnuje oblasti \acs{DSP} (\acl{DSP}), zejména jevům, které nastanou při nedodržení Nyquistovy podmínky pro \ac{symfvz}.%
%\footnote{Tato věta je pouze ukázkou použití příkazů pro sazbu zkratek.}
%
%Šablona je nastavena na \emph{dvoustranný tisk}.
%Nebuďte překvapeni, že ve vzniklém PDF jsou volné stránky.
%Je to proto, aby důležité stránky jako např.\ začátky kapitol začínaly po vytisknutí a svázání vždy na pravé straně.
%%
%Pokud máte nějaký závažný důvod sázet (a~zejména tisknout) jednostranně, nezapomeňte si přepnout volbu \texttt{twoside} na \texttt{oneside}!