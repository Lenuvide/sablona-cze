%7. Na vhodném příkladu odzkoušejte, ověřte a demonstrujte funkčnost celého systému.

\chapter{Ostetování měřícího systému}
nalevače
nevyužito. Byla i odstraně databáze s nalévači.

%Uvest sem tip na ověření systému

\section{Doba inventury po implementaci nového systému}

započítat čas i zápisu do papíru (než se najde položky muže to trvat)

Příklad funkčnosti celého systému jako celek - video v příloze

otestovat zatížení programu (podivat se do spravce uloh)
zkusit vymyslet nějaký testy, kde se něco časuje

\section{Ověření přesnosti měřícího systému jako celek}

Jak bylo uevedeno ve 3. kapitole do systému vstupuje několik chyb a i když jsme je teoreticky ověřili a spočítali celkovou přesnost měřícího systému pro stanovení přesnosti váhy, která byla jediným prvek ovlivňující výslednou přesnost je dobré si tuto přesnost ověřit i v praxi. Protože systém slouží pro měření destilátů, kde jejich hustota není nikde vedena, bude ověření provedeno experimentálně s odměrným válcem vyšší přesnosti a váhou, která stanový výslednou hodnotu objemu. Přesnost měřícího systému vyšla +-xx ml, proto pro ověření byl zvolen válec s přesností 0,05 ml na 250 ml, testovný destiláty jsou:

Válce jsou mále a nepojmou více kapaliny, proto tování přesnosti na vyšších hodnotách bud naskládáme více přesnějších válců na váhu nebo vypočítáme z něj hustotu a dopočítáme objem z vyšších hmotností