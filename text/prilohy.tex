

\chapter{Obsah elektronické přílohy}

1) Mám sem nahrát celý adresář programu? Včetně složek pro obrázky/ikony? Mám tyto složky odevzdat i s obrázky? 

2) Mám nahrát i databázové soubory .db? 

3) Mám do elektronické přilohy zařadit i dokumentaci ke krabičce nebo stačí ji zobrazit jen zde v příloze tohoto dokumentu?

\bigskip

{\small
%
\dirtree{%.
.1 /\DTcomment{kořenový adresář přiloženého archivu}.
.2 main.py.
.2 gui.py.
.2 database.py.
.2 scale.py.
.2 barcode\_sensor.py.
}
}

\bigskip
\bigskip
\bigskip

{\small
%
\dirtree{%.
.1 /\DTcomment{kořenový adresář přiloženého archivu}.
.2 firmware./\DTcomment{Složka fimwaru}.
.3 main.py.
.3 gui.py.
.3 database.py.
.3 scale.py.
.3 barcode\_sensor.py.
.3 database.db.
.3 inventory.db.
.3 Images /\DTcomment{Seznam složek obsahující obrázky}.
.4 Alcohol /\DTcomment{Obrázky destilátů}.
.4 Icons /\DTcomment{Ikony tlačítek}.
.2 Barcode scaner case documentation.xx.
}
}








%Elektronická příloha je často nedílnou součástí semestrální nebo závěrečné práce.
%Vkládá se do informačního systému VUT v~Brně ve vhodném formátu (ZIP, PDF\,\dots).
%
%Nezapomeňte uvést, co čtenář v~této příloze najde.
%Je vhodné okomentovat obsah každého adresáře, specifikovat, který soubor obsahuje důležitá nastavení, který soubor je určen ke spuštění, uvést nastavení kompilátoru atd.
%Také je dobře napsat, v~jaké verzi software byl kód testován (např.\ Matlab 2018b).
%Pokud bylo cílem práce vytvořit hardwarové zařízení,
%musí elektronická příloha obsahovat veškeré podklady pro výrobu (např.\ soubory s~návrhem DPS v~Eagle).
%
%Pokud je souborů hodně a jsou organizovány ve více složkách, je možné pro výpis adresářové struktury použít balíček \href{https://www.ctan.org/pkg/dirtree}{\texttt{dirtree}}.
%
%\bigskip

%{\small
%%
%\dirtree{%.
%.1 /\DTcomment{kořenový adresář přiloženého archivu}.
%.2 logo\DTcomment{loga školy a fakulty}.
%.3 BUT\_abbreviation\_color\_PANTONE\_EN.pdf.
%.3 BUT\_color\_PANTONE\_EN.pdf.
%.3 FEEC\_abbreviation\_color\_PANTONE\_EN.pdf.
%.3 FEKT\_zkratka\_barevne\_PANTONE\_CZ.pdf.
%.3 UTKO\_color\_PANTONE\_CZ.pdf.
%.3 UTKO\_color\_PANTONE\_EN.pdf.
%.3 VUT\_barevne\_PANTONE\_CZ.pdf.
%.3 VUT\_symbol\_barevne\_PANTONE\_CZ.pdf.
%.3 VUT\_zkratka\_barevne\_PANTONE\_CZ.pdf.
%.2 obrazky\DTcomment{ostatní obrázky}.
%.3 soucastky.png.
%.3 spoje.png.
%.3 ZlepseneWilsonovoZrcadloNPN.png.
%.3 ZlepseneWilsonovoZrcadloPNP.png.
%.2 pdf\DTcomment{pdf stránky generované informačním systémem}.
%.3 student-desky.pdf.
%.3 student-titulka.pdf.
%.3 student-zadani.pdf.
%.2 text\DTcomment{zdrojové textové soubory}.
%.3 literatura.tex.
%.3 prilohy.tex.
%.3 reseni.tex.
%.3 uvod.tex.
%.3 vysledky.tex.
%.3 zaver.tex.
%.3 zkratky.tex.
%%.2 navod-sablona\_FEKT.pdf\DTcomment{návod na používání šablony}.
%.2 sablona-obhaj.tex\DTcomment{hlavní soubor pro sazbu prezentace k~obhajobě}.
%%.2 readme.txt\DTcomment{soubor s~popisem obsahu CD}.
%.2 sablona-prace.tex\DTcomment{hlavní soubor pro sazbu kvalifikační práce}.
%.2 thesis.sty\DTcomment{balíček pro sazbu kvalifikačních prací}.
%}
%}

\chapter{Dokumentace krabičky čtečky čárových kódů}

Musí se dodělat

\chapter{Naměřená data}

\section{Naměřené hmotnosti prázdných lahví}

\section{Tabulka naměřených objemů neotevřených lahví}

\section{Průběhy nárustu hmotnosti z duvodu orosení po do 1h}

\chapter{Ostatní okna GUI}
\chapter{Příručka GUI}
