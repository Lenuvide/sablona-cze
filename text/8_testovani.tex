%7. Na vhodném příkladu odzkoušejte, ověřte a demonstrujte funkčnost celého systému.

\chapter{Otestování měřícího systému}


Sundání nalévače většinou zabralo 1 - 2 s a našroubování víčka

%Uvest sem tip na ověření systému

%%%%%%%%%%%%%%%%%%%%%%%%%%%%%%%%%%%%%%%%%%%%%%%%%%%%%%%%%%%%%%%%%%%%%%
\section{Doba inventury po implementaci nového systému}
Testování bylo rozděleno do dvou kategorií, v jednom jsem 

započítat čas i zápisu do papíru (než se najde položky muže to trvat)

Příklad funkčnosti celého systému jako celek - video v příloze

otestovat zatížení programu (podivat se do spravce uloh)
zkusit vymyslet nějaký testy, kde se něco časuje


\begin{table}[H]
    \centering
    \begin{tabular}{|c|c|}
    \hline
        Metoda & Čas [s]\\ \hline \hline
        Odměrné válce & x\\ \hline
        Navrhované zařízení & x\\ \hline
        %Měřicí systém & x\\ \hline
    \end{tabular}
    \caption{Porovnání dob trvání inventury provedené odměrnými válci s navrženým měřicím systémem}
    \label{tab:delka_inventury}
\end{table}

Pro měření ruzných objemu se u měření válcem na alkohol a váhou časy nemění

% Requires: \usepackage{multirow}

\begin{table}[H]
    \centering
    \newcolumntype{C}[1]{>{\centering\arraybackslash}m{#1}}
    \renewcommand{\arraystretch}{1.2}   % zvětší řádkování v tabulce
    \begin{tabular}{|c|C{2,7 cm}|C{2,7 cm}|C{2,7 cm}|}
        \hline
        & \multicolumn{3}{c|}{Čas [s]} \\ \hline
        \begin{tabular}[c]{@{}c@{}} Zbytkový objem \\ v lahvi [l] \end{tabular} & \begin{tabular}[c]{@{}c@{}} Odměrný \\ válec \end{tabular} & \begin{tabular}[c]{@{}c@{}} Odměrný válec \\ na alkohol \end{tabular} & \begin{tabular}[c]{@{}c@{}} Navrhované \\ zařízení \end{tabular} \\ \hline \hline
        0,1   &            &            &            \\ \hline
        0,5 &            & --         & --         \\ \hline
        1 &            & --         & --         \\ \hline
    \end{tabular}
    \caption{Časová náročnost měření zbytkového objemu v lahvi pomocí odměrných válců a pomocí navrženého měřicího systému}
    %\caption{Měření doby měření zbytkového objemu v lahvi pomocí odměrných válců a měřicího systému}
    \label{tab:cas_mereni}
\end{table}



%\begin{tabular}{|l|c|c|c|}
%  \hline
%  & \multicolumn{3}{c|}{Čas [s]} \\ \cline{2-4}
%  Objem lahve [l] & válec oby. & válec alc. & Váha \\ \hline
%  lahev (1 l)   & & & \\ \hline
%  lahev (0,5 l) & & -- & -- \\ \hline
%  lahev (0,1 l) & & -- & -- \\ \hline
%\end{tabular}

% Requires: \usepackage{multirow}

%\begin{table}
%    \centering
%    \begin{tabular}{|c|c|c|c|}
%        \hline
%        \multirow{2}{*}{Objem lahve [l]} & \multirow{2}{*}{Váha} & \multicolumn{2}{c|}{Čas [s]} \\ \cline{3-4}
%                                         &                       & válec oby. & válec alc. \\ \hline
%        lahev (1 l)                      & 3                     &            &            \\ \hline
%        lahev (0,5 l)                    & 4                     &            &            \\ \hline
%        lahev (0.1 l)                    & 5                     &            &            \\ \hline
%    \end{tabular}
%    \caption{Tabulka objemů lahví a časů.}
%    \label{tab:placeholdeeer}
%\end{table}


%%%%%%%%%%%%%%%%%%%%%%%%%%%%%%%%%%%%%%%%%%%%%%%%%%%%%%%%%%%%%%%%%%%%%%
\section{Ověření přesnosti měřícího systému jako celek}

Jak bylo uevedeno ve 3. kapitole do systému vstupuje několik chyb a i když jsme je teoreticky ověřili a spočítali celkovou přesnost měřícího systému pro stanovení přesnosti váhy, která byla jediným prvek ovlivňující výslednou přesnost je dobré si tuto přesnost ověřit i v praxi, reálným měřením. Protože systém slouží pro měření destilátů, kde jejich hustota není nikde vedena, bude ověření provedeno experimentálně s odměrným válcem vyšší přesnosti a váhou, která stanový výslednou hodnotu objemu. Přesnost měřícího systému vyšla +-xx ml, proto pro ověření byl zvolen válec s přesností 0,05 ml na 250 ml, testovný destiláty jsou:

Válce jsou mále a nepojmou více kapaliny, proto tování přesnosti na vyšších hodnotách bud naskládáme více přesnějších válců na váhu nebo vypočítáme z něj hustotu a dopočítáme objem z vyšších hmotností

Součástí přílohy je video znázorňující průběh inventury/vážení lahví

Testování bylo provedeno pro 3 lahve různých velikostí: 0,5l, 0,7l a 1l, testování dále bylo provedeno pro různé hladiny