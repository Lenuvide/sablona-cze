\chapter{Inventura}
\label{inventura}
%\section{Definice}

V ekonomice se pod pojmy inventura rozumí zvláštní administrativní činnost, při které se k určitému datu zjišťuje skutečný stav majetku a jestli tento stav odpovídá se stavem majetku v účetnictví. 
\cite{Zákon o účetnictví}
%Stav majetku se zjistí jako rozdíl dvou po sobě jdoucích inventur. V praxi se spočítá rozdíl SS a SvU ze dvou po sobě jdoucích inventur

V praxi to znamená, že se porovnají rozdíly skutečného stavu majetku(např. peněžní hodnota prodaného zboží) a stavu majetku v účetnictví(např. stav kasy) ze dvou po sobě jdoucích inventur, tedy poslední a předposlední inventura.

%sdsdasdasdasd

V našem případě se budeme zabývat fyzickou inventurou pro HoReCa podniky. %\cite{Zákon o účetnictví}
%Zdroj: https://www.zakonyprolidi.cz/cs/1991-563#cast5

%\section{Význam}
%%Pracovní a stanovené váhy jako měřidlo pro inventurní účely
\section{Váhy z hlediska legislativy}
Námi navrhovaný měřicí systém obsahuje váhu pro měření hmotnosti zbytkové kapaliny v láhvi. Je nutné dodržet veškeré legislativní předpisy stanovené zákonem č. 505/1990 Sb. o metrologii §3 pro používání vah v obchodním styku.
Váhy se dále podle zákona dělí na pracovní a stanovená měřidla. \cite{použití elektronických vah v obchodním styku}
%Zdroj: https://www.unmz.cz/files/metrologie/v%C3%BDstupy%20z%20PRM/uvv-vii-20-18-obchod-web.pdf

\subsection{Stanovené měřidla}
Stanovené měřidlo je vysokopřesné měřící zařízení používané primárně pro kalibraci nebo ověřování jiných měřicích přístrojů. Jeho hlavním účelem je poskytovat referenční standard, proti kterému se porovnávají ostatní měřidla. Využívají se v obchodním odvětví, kde mají přímý vliv na koncového spotřebitele. Zákonem jsou definována:
%Vygenerovano AI, překopat

\textit{Stanovená měřidla jsou měřidla, která Ministerstvo průmyslu a obchodu (dále jen "ministerstvo") stanoví vyhláškou k povinnému ověřování s ohledem na jejich význam:}

\textit{a) v závazkových vztazích, například při prodeji, nájmu nebo darování věci, při poskytování služeb nebo při určení výše náhrady škody, popřípadě jiné majetkové újmy,}

\textit{b) pro stanovení sankcí, poplatků, tarifů a daní,}

\textit{c) pro ochranu zdraví,}

\textit{d) pro ochranu životního prostředí,}

\textit{e) pro bezpečnost při práci, nebo}

\textit{f) při ochraně jiných veřejných zájmů chráněných zvláštními právními předpisy.} \cite{Zákon o metrologii}
%Zdroj: https://www.zakonyprolidi.cz/cs/1990-505#cast1

\subsection{Pracovní měřidla}
Pracovní měřidla jsou běžně používána v průmyslových a výrobních procesech nebo domácnostech pro rutinní měření. Slouží k provádění měření v rámci každodenních operací. Zákonem jsou definována:
%Vygenerovano AI, překopat

\textit{Pracovní měřidla jsou měřidla, která nejsou etalonem ani stanoveným měřidlem.} \cite{Zákon o metrologii}
%Zdroj: https://www.zakonyprolidi.cz/cs/1990-505#cast1

%Rozebereme si co je pracovní a stanovená váha a rozdíli mezi nimi.

\section{Váha jako pracovní měřidlo}
\label{meridlo}
Podle výše citovaného zákonu se váha bude pohybovat v obchodním odvětví, ale nebude mít přímý vliv na koncového spotřebitele, tudíž váha by byla využívána pro interní činnosti podniku, kde závisí na majiteli, zda požaduje váhu jako stanovené měřidlo či nikoli. Pro naše účely tedy stačí i pracovní měřidlo.


%\section{Fyzická/hmotnostní inventura}
%\section{Směrnice} %Legislativa

%I když je inventura vnitří záležitost podniku, stále pro ní platí 
%Pro inventuru platí zákon