\chapter{Inventura v HoReCa provozech}
\label{inventura}
%\section{Definice}

V ekonomice se pod pojmy inventura rozumí zvláštní administrativní činnost, při které se k určitému datu zjišťuje skutečný stav majetku a jestli tento stav odpovídá se stavem majetku v účetnictví. 
\cite{Zákon o účetnictví}
%Stav majetku se zjistí jako rozdíl dvou po sobě jdoucích inventur. V praxi se spočítá rozdíl SS a SvU ze dvou po sobě jdoucích inventur

V praxi to znamená, že se porovnají rozdíly skutečného stavu majetku(např. peněžní hodnota prodaného zboží) a stavu majetku v účetnictví(např. stav kasy) ze dvou po sobě jdoucích inventur, tedy poslední a předposlední inventura.

\section{Limity přesného měření v provozu}

V provozovnách HoReCa je při inventarizaci zásob alkoholu kladen důraz na rychlost. Inventura se často provádí po skončení provozu nebo při střídání směn, kdy je potřeba během krátkého času zkontrolovat stav zásob. Cílem je zachytit výrazné nesrovnalosti, nikoli zajistit naprosto přesné změření každé lahve do posledního mililitru. V praxi to znamená, že obsluha zpravidla počítá plné lahve a u otevřených lahví odhaduje zbytkový objem (například vizuálně v centilitrech či jako podíl lahve - běžně u vysoce viskózních destilátů, kde přelévání přináší velké ztráty). Tento postup je podstatně rychlejší než každou láhev přesně měřit a umožňuje odhalit zásadní rozdíly (typicky manko svědčící o krádeži či zapomenuté tržbě), aniž by personál trávil neúměrně dlouhý čas měřením - v opačném případě by náklady na vykonání detailní inventury byly nákladnější než samotná chyba fyzického stavu.
%V takovém případě by se ani zaměstnavateli nevyplatilo.. i kdyby kompenzovaná

Moderní systémy skladové evidence dokonce nabízejí režimy “rychlé inventury”, které jsou navrženy pro bleskovou kontrolu klíčových položek během pár minut. To vše odráží fakt, že inventura slouží hlavně jako kontrolní mechanismus – má odhalit zjevné anomálie a větší rozdíly, zatímco drobné odchylky jsou v každodenním provozu tolerovány.
	
%Při inventurách, obzvlášť evidenci zbytkového objemu destilátů je kladen důraz na rychlost a efektivitu provedení. Cílem je zachytit výrazné nesrovnalosti jako krádež v podniku ne odměřit zbytkový objem na mililitry. Důležité je spočítat přesný stav kusového zboží, tedy neotevřených lahví ve skladu, při odměření zbytkového objemu se jedná jen o hrubý odhad, zda neubilo pulku lahve alkoholu. Aby bylo možné se co nejvíce přiblížit fyzického stavu s účetním stavem podníku bylo by nutné eliminovat chyby v pracovním provozu a při provádění inventury, což je v obou případech nereálné nebo vysoce nepraktické 
%
%Duvodem je 
%\begin{itemize}
%    \item Nedodržení risky na panáku - ve spěchu přelijeme
%    \item Rozlévání chlazeného alkoholu (menší objem) - ryska je dimenzovaná pro 20 °C
%    \item U alkoholu s nalévačem dochází k zvětrávání a vypařování ethanolu
%    \item samotné chyby při inventůře viz kapitola výše
%\end{itemize}
%
%Zaměstnavatelé sami nechtějí investovat peníze do dražšího zařízení nebo platit zaměstnance, protože čas navíc co stráví při inventuře je nákladnější než chyba měření pár millilitrů. 
%
%že v průběhu chodu podniku dochází kontinuálně ke ztrátam alkoholu, i když jen drobným jako podávání chlazeného alkholu (menší objem) - riska na panaku je dimenzovana pro 20°C, ve spěchu risku přelijeme, u alkoholu s nalévačem dochází k zvětrávání a vypařování ethanolu, při samotné inventůře si nemůžem dovolit čekat než lahev se zahřeje na pokojovou teplotu, tudíž "nikdy" stav fyzického stavu nebude sedět se stavem v účetnictví vždy se bude jednat o drobný rozdíli s kterýma musíme počítat, proto nemá cenu ani dělat velice přesnou inventarizaci.
%	
v HoReCa podnicích se tedy vysoká přesnost měření u inventur lihovin neuplatňuje zejména proto, že náklady (časové i materiální) na dosažení laboratorní přesnosti by převýšily přínosy. Přijatelná nepřesnost je vyvážena tím, že inventura spolehlivě zachytí podstatné odchylky, aniž by brzdila provoz, např. čekáním, než se vychlazená lahev zahřeje na pokojovou teplotu.

%– což je v souladu se současnou praxí v pohostinství a interními kontrolními postupy podniků.
%
%\cite{Zákon o účetnictví}
%Zdroj: https://www.zakonyprolidi.cz/cs/1991-563#cast5

\section{Metrologické postupy a legislativa inventury}
Z pohledu českého zákona o metrologii (č. 505/1990 Sb.) se inventura zásob prováděná pro interní potřeby podniku nepovažuje za měření pro obchodní styk. Podnik proto nemusí používat schválená ("cejchovaná") měřidla ani dodržovat postupy stanovené pro úřední měření. Má tak volnost zvolit metodu, která je pro daný provoz nejpraktičtější, byť by vykazovala vyšší měřicí nejistotu. V HoReCa provozech se proto běžně využívají odměrné válce nižší třídy přesnosti z důvodu lepší cenové dostupnosti.

\subsection{Váhy z hlediska legislativy}
Navrhovaný měřicí systém obsahuje váhu pro měření hmotnosti zbytkové kapaliny v láhvi. Dle výše zmíněného zákona o metrologii, konkrétně §3, se váhy dělí na:
\begin{itemize}
    \item \textbf{Stanovené měřidla} - je vysokopřesné měřící zařízení používané primárně pro kalibraci nebo ověřování jiných měřicích přístrojů. Jeho hlavním účelem je poskytovat referenční standard, proti kterému se porovnávají ostatní měřidla. Využívají se v obchodním odvětví, kde mají přímý vliv na koncového spotřebitele. \cite{Zákon o metrologii}
    \item \textbf{Pracovní měřidla} - Pracovní měřidla jsou běžně používána v průmyslových a výrobních procesech nebo domácnostech pro rutinní měření. Slouží k provádění měření v rámci každodenních operací. \cite{Zákon o metrologii}
\end{itemize}
\smallskip
Navrhovaný měřicí systém se bude pohybovat v obchodním odvětví, ale nebude mít přímý vliv na koncového spotřebitele, tudíž váha by byla využívána pro interní činnosti podniku, kde závisí na majiteli, zda požaduje váhu jako stanovené měřidlo, či nikoli. Pro účely navrhovaného systému stačí i pracovní měřidlo.\cite{použití elektronických vah v obchodním styku}

